%!TEX root = /Users/kevin/SkyDrive/KTH Work/Period 4 2014/GNSS/Labs/L3 - Computation of receiver position from code and phase measurements/Report/Lab 3.tex

\section{Introduction} 

% (fold)
\label{sec:introduction}  The task is to compute the coordinates of a GPS receiver at given time using P1-code pseudo ranges. My specific observation epoch is: 2004-02-02, 1 h 14 min 00 sec. The calculations were done with observation files and navigation files in RINEX format. The computation of this position is difficult because the satellite and receiver clocks are not synchronised, the satellite and receiver are both moving with varying velocities, and propagation times. There are many more problems that need to be accounted for like ionospheric and tropospheric corrections, but we have neglected them in this lab because we want to gain a general understanding before we take on the harder tasks
% section introduction (end)

