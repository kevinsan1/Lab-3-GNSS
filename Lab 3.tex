\documentclass[a4paper,11pt]{article}

% (fold)
\usepackage{booktabs,amsmath,multirow} 
\usepackage{mathtools} % needed for next line
\mathtoolsset{showonlyrefs}
\usepackage{fullpage}
 % only shows number of referenced equations
% \newcommand{\dpder}[3][]{\dfrac{\partial^{#1}#2}{\partial #3}} \addtolength{\oddsidemargin}{-.875in} \addtolength{\evensidemargin}{-.875in} \addtolength{\textwidth}{1.75in}
% \addtolength{\topmargin}{-.875in} \addtolength{\textheight}{1.75in}

% need for subequations
\usepackage{graphicx} 

% need for figures
\usepackage{verbatim}

% useful for program l,istings
\usepackage{color} 

% use if color is used in text
\usepackage{subfigure}

% use for side-by-side figures
\usepackage[pdfpagemode={UseOutlines},bookmarks=true,bookmarksopen=true, bookmarksopenlevel=0,bookmarksnumbered=true,hypertexnames=false, colorlinks,linkcolor={blue},citecolor={blue},urlcolor={red}, pdfstartview={FitV},unicode,breaklinks=true]{hyperref} 

% use for hypertext links, including those to external documents and URLs
\newcommand{\ajp}{AJP} 
\newcommand{\ig}{ 
\includegraphics} 
\newcommand{\tu}{\textup} \hypersetup{urlcolor=blue, colorlinks=true} 

% Colors hyperlinks in blue - change to black if annoying
% don't need the following. simply use defaults
\setlength{\baselineskip}{16.0pt} 

% 16 pt usual spacing between lines
\begin{comment}
	\pagestyle{empty} 
	
	% use if page numbers not wanted
\end{comment}

% Specifies the directory where pictures are stored
% above is the preamble (end)
\def\MyWidth{0.71\textwidth} % textwidth for multiple graphs
\begin{document} 
\title{Lab 3 - Computation of receiver position from code and phase measurements} 
\author{Kevin Mead}

%\affiliation{KTH Royal Institute of Technology, Electrotechnical Modelling} \email{kmead@kth.se}
\date{\today} 
\maketitle 
%!TEX root = /Users/kevin/SkyDrive/KTH Work/Period 4 2014/GNSS/Labs/L3 - Computation of receiver position from code and phase measurements/Report/Lab 3.tex

\section{Introduction} 

% (fold)
\label{sec:introduction}  The task is to compute the coordinates of receiver ROV at given time using P1-code  and L1 phase observations from two GPS receivers: REF and ROV. The receiver REF has known coordinates. The calculations were done with observation files and other values given in the Excel document \emph{DiffAssignmentL1.xls}. Unlike the last lab, where the satellite and receiver clocks were not synchronized, this lab synchronizes clocks of all the satellites to the system time.
% section introduction (end)


%!TEX root = /Users/kevin/SkyDrive/KTH Work/Period 4 2014/GNSS/Labs/L3 - Computation of receiver position from code and phase measurements/Report/Lab 3.tex

\section{Methodology} 

% (fold)
\label{sec:methodology}

% Describe the methods and instruments used to solve the problem. 
Matlab was used to solve for the following parameters and all of the following equations were taken from the technical report: GPS single point positioning algorithm.\cite{milanGPS} 

The steps and equations from this algorithm are documented in the Matlab code, Section~\ref{sec:main_code}. The pseudo ranges and navigation messages are used to calculate a more precise position of the receiver. The shift in position is a measure of the time for the signal to travel from satellite to receiver. 

This time traveled by the signal multiplied by the speed of light is the pseudo range. 

All the satellites were used in this calculation in order to get the best possible approximation of the receiver coordinates. Since we will be using least-squares method to solve for the parameters, the more independent equations we have, the better the approximation will be. 

First, the signal propagation time was calculated using, 
\begin{equation*}
	\Delta t_A^s = P_A^s(\tilde{t_A})/c 
\end{equation*}
which was then used to calculate the nominal transmission time, 
\begin{equation*}
	\tilde{t}^s = \tilde{t_A} - P_A^s(\tilde{t_A})/c 
\end{equation*}
where $\tilde{t_A}$ is the nominal time of signal reception by the receiver. For this lab report, $\tilde{t_A}=90840~$seconds, or 2 days, 1 hour, and 14 minutes. The weeks in the year was omitted because the difference between epoch times was less than a day and therefore would have cancelled out.

Next the satellite clock correction, $\delta t_{L1}^s$, was computed using equations, 
\begin{align*}
	\delta t_{L1}^s &= \Delta t_{SV}-T_{GD} \\
	\Delta t_{SV} &= a_{f0} + a_{f1}(\tilde{t}^s - t_{oc}) + a_{f2}(\tilde{t}^s - t_{oc})^2 + \Delta t_r 
\end{align*}
but the relativistic correction, $\Delta t_r = Fe\sqrt{A} \sin(E_k)$ was neglected until the eccentric anomaly and satellite coordinates were computed. Next, the system time of signal transmission, $t^s$ was calculated using the satellite clock correction. The eccentric anomaly was calculated using Kepler's equation, 
\begin{equation*}
	E_k = M_k + e \sin(E_k) 
\end{equation*}
and the solution was iterated until convergence. The relativistic correction is now calculated and $t^s$ is corrected in order to compute the satellite coordinates at that time given by the equations, 
\begin{align}
	X^s &=x'_k \cos \Omega_k -y'_k \cos i_k \sin \Omega_k \nonumber \\
	Y^s &=x'_k \sin \Omega_k +y'_k \cos i_k \cos \Omega_k \nonumber \\
	Z^s &= y'_k \sin i_k \nonumber 
\end{align}
Now that the satellite coordinates have been calculated, the satellite clock correction is corrected again to include the relativistic correction, $\Delta t_r$. Now the approximate distance of the satellite is calculated using the equation, 
\begin{equation*}
	\rho _{\text{A0}}^s\left(\tilde{t}_A\right) = \sqrt{\left( X^s-X_{\text{A0}}+\dot{\Omega}_eY_{\text{A0}} \text{$\Delta $t}^s \right)^2+\left( Y^s-Y_{\text{A0}}-\dot{\Omega}_eX_{\text{A0}} \text{$\Delta $t}^s \right)^2+\left( Z^s-Z_{\text{A0}} \right)^2} 
\end{equation*}

The calculation of these parameters is done for all satellites we are interested in. Since we are computing four unknown parameters, the satellite lower limit is also four. More than four should only increase our accuracy assuming they aren't outliers. The result will be used to get more accurate coordinates for the receiver. 

The least squares method is used to calculate the corrections to the receiver position and system time of signal reception. Using values from different satellites, we solve, 
\begin{equation*}
	\mathbf{L} - \mathbf{v} = \mathbf{AX} 
\end{equation*}
where $\mathbf{v}$ is a vector of unknown residuals, and the vector $\mathbf{X}$ is solved for with the LSQ solution, 
\begin{equation*}
	\mathbf{X} = (\mathbf{A}^\text{T} \mathbf{A})^{-1} \mathbf{A}^\text{T} \mathbf{L} = [\Delta X~~\Delta Y~~\Delta Z~~\text{c} \delta \text{t}_\text{A} ]^\text{T}
\end{equation*}
where $\mathbf{A}$ and $\mathbf{L}$ represent 
\begin{align*}
	\mathbf{A} &= 
	\begin{bmatrix}
		a_x^{(1)} & a_y^{(1)} & a_z^{(1)} &1 \\
		\vdots &\vdots &\vdots &\vdots \\
		a_x^{(n)} & a_y^{(n)} & a_z^{(n)} &1 \\
	\end{bmatrix}
	\\
	\mathbf{L} &= 
	\begin{bmatrix}
		P_A(\tilde{t}_A)-\rho_{A0}^{(1)} + c \delta t_{L1}^1\\
		\vdots \\
		P_A(\tilde{t}_A)-\rho_{A0}^{(n)} + c \delta t_{L1}^n\\
	\end{bmatrix}
\end{align*}

The new estimated coordinates are computed by adding onto the approximate receiver position given in the observation file. 
\begin{align*}
	X_A &= X_{A0} + \Delta X \\
	Y_A &= Y_{A0} + \Delta Y \\
	Z_A &= Z_{A0} + \Delta Z \\
\end{align*}

Now the calculation of $\rho _{\text{A0}}^s\left(\tilde{t}_A\right)$, $\mathbf{L}$, $\mathbf{A}$ is done again using the updated receiver coordinates until the solution has converged.  We say that the solution has converged if the following condition is fulfilled: $|(\text{v}^{\text{T}} \text{v})_i - (\text{v}^{\text{T}} \text{v})_{i-1}|< 1e-5$.  The results fulfilling the convergence condition are given in Table~\ref{tab:receiverCoordinates}.

% section methodology (end)

%!TEX root = /Users/kevin/SkyDrive/KTH Work/Period 4 2014/GNSS/Labs/L3 - Computation of receiver position from code and phase measurements/Report/Lab 3.tex

\section{Results} % (fold)
\label{sec:results}

% Present results from all steps you performed in the exercise. Images or tabular results should be included in this section. Describe and analyse the results and discuss the meaning and implications of the results. Answer the questions written in the lab instructions. Pay attention to the quality of input data and their significance for interpreting the reliability of results/conclusions.
The results for the LSQ of equations is,
\begin{equation}
	\mathbf{X}(m) = \left[\begin{matrix}
			$-0.03	$						\\	
			$0.16	$						\\  	
			$0.22	$						\\  	
			$951566.01$						\\   	
			$1014451.16$					\\   	
			$-1138240.42$					\\  	
			$-9347.42$						\\  	
			$-1078921.56$					\\  	
		\end{matrix}\right]
	\label{eq:xMatrix}
\end{equation} % (eq:xmatrix)
with $\mathbf{L}$ and $\mathbf{A}$ equal to,
\begin{equation}
	\begin{array}{cc}
		\mathbf{L}(m)= \left[\begin{matrix}
				$181077.04$ \\	
				$193043.66$	\\  	
				$-216600.08$\\  	
				$-1778.67$	\\   	
				$-205312.07$\\   	
				$0.13$		\\  	
				$-0.22$		\\  	
				$-0.09$		\\  	
			\end{matrix}\right] & 
			\mathbf{A}(m)= \left[\begin{matrix}
     $ -1.13$ & $ 0.25$ & $-0.08 $ & 0.19 &    0 &    0 &    0 &    0\\
     $-0.199$ & $ 0.58$ & $ -0.31$ &    0 & 0.19 &    0 &    0 &    0\\
     $ 0.306$ & $-0.30$ & $ -0.34$ &    0 &    0 & 0.19 &    0 &    0\\
     $-0.605$ & $ 0.52$ & $ -0.10$ &    0 &    0 &    0 & 0.19 &    0\\
     $-0.338$ & $-0.84$ & $-0.049$ &    0 &    0 &    0 &    0 & 0.19\\
     $ -1.13$ & $ 0.25$ & $-0.08 $ &    0 &    0 &    0 &    0 &    0\\
     $-0.199$ & $ 0.58$ & $ -0.31$ &    0 &    0 &    0 &    0 &    0\\
     $ 0.306$ & $-0.30$ & $ -0.34$ &    0 &    0 &    0 &    0 &    0\\
     $-0.605$ & $ 0.52$ & $ -0.10$ &    0 &    0 &    0 &    0 &    0\\
     $-0.338$ & $-0.84$ & $-0.049$ &    0 &    0 &    0 &    0 &    0\\
\end{matrix}\right]
	\end{array}
\end{equation} % (eq:lmatrixAmatrix)
The double differences were computed:
\begin{equation}
	\begin{array}{cc}
		\lambda \varphi = 
		\left[\begin{matrix}
				$180848.77$ \\
				$192567.29$ \\
				$-216415.03$\\
				$-2197.22$\\
				$-204701.52$\\
		\end{matrix}\right]
		 &
		P =  
		\left[\begin{matrix}
			$-228.14$\\
			$-476.60$\\
			$184.96$\\
			$-418.71$\\
			$610.13$\\
		\end{matrix}\right]
	\end{array}
\end{equation}
The results agree completely with the reference material.


\begin{table}[h]
	\begin{center}
		\begin{tabular}{llr}
		\toprule
		\multicolumn{2}{c}{Item} \\
		\cmidrule(r){1-2}
		Animal    & Description & Price (\$) \\
		\midrule
		Gnat      & per gram    & 13.65      \\
		          & each        & 0.01       \\
		Gnu       & stuffed     & 92.50      \\
		Emu       & stuffed     & 33.33      \\
		Armadillo & frozen      & 8.99       \\
		\bottomrule
		\end{tabular}
	\end{center}
\end{table}
% section results (end)

%!TEX root = /Users/kevin/SkyDrive/KTH Work/Period 4 2014/GNSS/Labs/L3 - Computation of receiver position from code and phase measurements/Report/Lab 3.tex


\section{Analysis and Discussion} % (fold)
\label{sec:analysis_and_discussion}
% In this part you should discuss your results by, for example, considering the following questions:
\begin{itemize}
	\item Are the results reasonable? Compare the results with your expectations.
	
	Yes the results are reasonable.  The final position of the receiver is very close to the approximate position and this is shown in Table~\ref{tab:receiverCoordinates}.  We also get very consistent results if we compute the same calculations for different epochs.
	\item Can we draw any conclusion/implications from the results?  
	
	We can conclude that the approximate coordinates are very close to the true receiver position, even though we did not factor in ionospheric and tropospheric effects.  
	\item Are results reliable and accurate?
	
	I believe the results are reliable and accurate because the $\sigma$ value is very small for all x,y,z and all 11 satellites were taken into account.  But the least squares method can give poor results if there are any outliers or the errors aren't Gaussian distributed.\cite{Garcia:1999:NMP:554354}  Looking at other epochs in the SPP Results, we can see that they also give approximately the same position for the receiver.
	\item Would it have been more appropriate to use another method? Does the method need to be further developed?
	
	I think it is only necessary to further develop this method if you needed a higher degree of accuracy.  This method could be further developed to take into account the ionospheric and tropospheric effects and factor in all the given epochs for the satellites.  I think it could me more appropriate to use the difference method is best used when you have more satellites and tropospheric and ionospheric effects are cancelled out in the equations.  This means our final solution should be more accurate using the difference method.
\end{itemize}

% section analysis_and_discussion (end)

\bibliographystyle{IEEEtran}
\bibliography{My_Book}
%!TEX root = /Users/kevin/SkyDrive/KTH Work/Period 4 2014/GNSS/Labs/L3 - Computation of receiver position from code and phase measurements/Report/Lab 3.tex


  \section{Main Code} % (fold)
  \label{sec:main_code}

    
\subsection*{Contents}

\begin{itemize}
\setlength{\itemsep}{-1ex}
   \item Lab 3
   \item Import file
   \item Constants
   \item Variables
   \item Main steps of Calculation
   \item 1. Synchronize observables from both receivers using equations
   \item 2. Compute single and double differences - equations (15) and
   \item 3. Compute coefficients ax , ay , az and rhoAB\_0
   \item 4. Fill in matrixes A, L and compute least square solution of
\end{itemize}


\subsection*{Lab 3}

\begin{par}
Difference Equations for GPS
\end{par} \vspace{1em}
\begin{verbatim}
clear all; clc;
\end{verbatim}


\subsection*{Import file}

\begin{verbatim}
addpath(['/Users/kevin/SkyDrive/KTH Work/'...
    'Period 4 2014/GNSS/Labs/'...
    'L3 - Computation of receiver position '...
    'from code and phase measurements'])
excfile = xlsread('DiffAssignmentL1');
excfile = padarray(excfile,5,'pre');
cellDiff = num2cell(excfile); % same as a cell
\end{verbatim}


\subsection*{Constants}

\begin{verbatim}
c = 299792458; % speed of light (m/s)
[lambda1,lambda2] = cellDiff{78:79,2};
mu = 3.986005e14; % universal gravitational parameter (m/s)^3
omega_e_dot = 7.2921151467e-5; % earth rotation rate (rad/s)
F = -4.442807633e-10; % s/m^1/2
nRows = 5;
\end{verbatim}


\subsection*{Variables}

\begin{verbatim}
dtA = excfile(24,3);
dtB = excfile(25,3);
sNum = excfile((28:33),1);
refSatNum = 20;
for i = 1:length(sNum) % vector of ones at satellite numbers
    vectOnes(sNum(i),1) = 1;
end
[Xref,Xrov,Yref,Yrov,Zref,Zrov] = cellDiff{(37:38),3:5};
for i = 43:48 %
    rhoDot(excfile(i-15,1),1) = excfile(i-15,2);
    satCoordinates(excfile(i,1),:) = excfile(i,3:5);
    rhoAtoP(excfile(i+23,1),:) = excfile(i+23,2);
    rhoB0toP(excfile(i+23,1),:) = excfile(i+23,3);
end
count = 1;
for i = 6:3:21
   P1ref(sNum(count),1) = excfile(i,3);
   P1rov(sNum(count),1) = excfile(i+1,3);
   phi1ref(sNum(count),1) = excfile(i,4);
   phi1rov(sNum(count),1) = excfile(i+1,4);
   count = count + 1;
end
weightMatrix = excfile(52:61,1:10);
\end{verbatim}


\subsection*{Main steps of Calculation}



\subsection*{1. Synchronize observables from both receivers using equations (14)}

\begin{verbatim}
P1refeq14 = P1ref + rhoDot*dtA;
P1roveq14 = P1rov + rhoDot*dtB;
phi1refeq14 = lambda1*phi1ref + rhoDot * dtA;
phi1roveq14 = lambda1*phi1rov + rhoDot * dtB;
\end{verbatim}


\subsection*{2. Compute single and double differences - equations (15) and (18). Use satellite 20 as a reference satellite for double differencing.}


\begin{verbatim}
P1eq15 = P1refeq14 - P1roveq14;
phi1eq15 = (phi1refeq14 - phi1roveq14);
dtAB = dtB - dtA;
% Double differences
phi1eq18 = phi1eq15 - phi1eq15(refSatNum)*vectOnes;
P1eq18 = P1eq15 - P1eq15(refSatNum)*vectOnes;
\end{verbatim}


\subsection*{3. Compute coefficients $a_X$ , $a_Y$ , $a_Z$ and $rho_{AB,0}^{pq}$ - equations (12).}

\begin{verbatim}
ax_S_rov = -1 * (satCoordinates(:,1)-Xrov*vectOnes)./(rhoB0toP);
ay_S_rov = -1 * (satCoordinates(:,2)-Yrov*vectOnes)./(rhoB0toP);
az_S_rov = -1 * (satCoordinates(:,3)-Zrov*vectOnes)./(rhoB0toP);
aXB_st = -ax_S_rov + ax_S_rov(20);
aYB_st = -ay_S_rov + ay_S_rov(20);
aZB_st = -az_S_rov + az_S_rov(20);
rhoAB_0 = rhoAtoP - rhoB0toP;
\end{verbatim}


\subsection*{4. Fill in matrixes A, L and compute least square solution of}

\begin{par}
equations (21).
\end{par} \vspace{1em}
\begin{verbatim}
p_AB_s = rhoB0toP-rhoAtoP;
nv = [4,5,6,24,25];
A = [aXB_st(nv,1), aYB_st(nv,1), aZB_st(nv,1)];
A = [ A , eye(length(nv)) * lambda1 ; A , zeros(length(nv)) ];
% L
L = [phi1eq18(nv) - rhoAB_0(nv) + ones(5,1)*rhoAB_0(20);...
    P1eq18(nv) - rhoAB_0(nv) + ones(5,1)*rhoAB_0(20)];
%
X = inv(A'*weightMatrix*A)*A'*weightMatrix*L;
\end{verbatim}

  
  % section main_code (end)  

    

%\noindent Updated \today.} 
\end{document} 
